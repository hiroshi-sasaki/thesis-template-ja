\chapter{提案手法}\label{ch:proposedsytem}

提案手法について説明する。
図を用いると効果的なことが多いので積極的に活用する。
図にはベクタイメージを用いた方が綺麗なのでできる限りPDFファイルを用いること\footnote{\cref{sec:results}で登場するグラフに関しても同様である。}。
なおPDFファイルの上下左右に余白があると思ったような見た目にならないので、例えば下記のコマンドなどで余白を取り除くと良い。

\begin{lstlisting}[language=bash,escapeinside={(*}{*)}]
(*\colorbox{gray90}{\% pdfcrop --margins 0 input.pdf output.pdf}*)
\end{lstlisting}

\begin{figure}[ht]
  \centering
  \includegraphics[width=0.5\textwidth]{examples/figures/square}
  \caption{正方形}\label{fig:square}
\end{figure}

図は必ず本文中で参照すること。
\cref{fig:square}は正方形である。

\begin{figure}[ht]
  \centering
  \begin{subfigure}[t]{0.45\textwidth}
  \centering
    \includegraphics[width=0.8\linewidth]{examples/figures/square}
    \caption{正方形}\label{subfig:square}
  \end{subfigure}
  \quad
  \begin{subfigure}[t]{0.45\textwidth}
  \centering
    \includegraphics[width=0.8\linewidth]{examples/figures/circle}
    \caption{円}\label{subfig:circle}
  \end{subfigure}
  \caption{正方形と円}\label{fig:square-circle}
\end{figure}

\cref{fig:square-circle}は正方形と円で、\cref{subfig:circle}は円である。
